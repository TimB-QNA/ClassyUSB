\chapter{Introduction}
The most pertinent question when writing a new library is why. For USB devices this is very simple. Existing libraries are highly complex, appallingly documented, and require a PhD in software engineering to add or remove components. That's ignoring the fact that many are tied to vendors, and as such the bloat is appalling, as every device has to be supported. Great for quickly throwing some code together, as a proof of concept, rubbish for actually building kit which is maintainable. For professional work, I wouldn't use the Arduino libraries, it's a good introduction to microcontrollers but it's probably not the solution you're looking for long-term.

This library attempts to fix some of those problems. There are always some tradeoffs, and its design favours ease of use over efficiency. Currently not many platforms are supported, but adding platform support is a matter of inheriting one class and writing the hardware support; everything else remains the same.

\section{Supported Platforms}

Only one platform is currently supported.

\begin{longtable}{ c c c }
Manufacturer & IC & Core Technology \\
\endhead
Atmel (Microchip) & SAMD21 & ARM Cortex M0 \\
& & \\
\caption{Supported microcontrollers}
\label{table:supportedMicrocontrollers}
\end{longtable}

\section{Limitations}
There as some limitations with the way this library is built. In most cases they will be minor.

\subsection{Devices are composite.}
This isn't a major limitation, but it's something to be aware of as it means that the descriptors tend to be longer, and therefore more memory is required for the primary endpoint.

\subsection{Hardware has limits}
Microcontrollers have a limited number of endpoints, often arranged as endpoint pairs. At present the code does not try to use these pairs intelligently, so you only get half the endpoints you would expect. This will limit the number of components you can have.

This will be addressed at some point in the future.

\subsection{Definitions for code tuning}
The code can be tuned by means of the definitions below. This is a self-imposed limitation, but the ability to tune the memory usage may be very helpful.

\begin{itemize}
  \item USB\_MAX\_COMPONENT\_ENDPOINTS
    \subitem Default 3
    \subitem Maximum number of endpoints allowed in and component or subcomponent.

  \item PLATFORM\_ENDIAN
    \subitem Default 0
    \subitem Platform endian. Little (0) or Big (1). No conversion is carried out if the platform endian matches USB.

  \item USB\_MAX\_COMPONENTS
    \subitem Default 3
    \subitem Maximum number of components in the device. Can be reduced to save memory.

  \item USB\_MAX\_ENDPOINTS
    \subitem Default 8
    \subitem Maximum number of endpoints handled. Can be reduced to save memory, may need to be increased for complex devices.

  \item USB\_MAX\_STRINGDESCRIPTORS
    \subitem Default 8
    \subitem Number of available string descriptors. Strings can use memory quickly so limit this where possible.
\end{itemize}
